% resume.tex
% vim:set ft=tex spell:

\documentclass[10pt,letterpaper]{article}
\usepackage[letterpaper,margin=0.75in]{geometry}
\usepackage[utf8]{inputenc}
\usepackage{mdwlist}
\usepackage[T1]{fontenc}
\usepackage{textcomp}
\usepackage{tgpagella}
% \usepackage{xcolor}
\usepackage{hyperref}
\hypersetup{colorlinks=false}%,linkbordercolor=red,linkcolor=green,pdfborderstyle={/S/U/W 1}}

\pagestyle{empty}
\setlength{\tabcolsep}{0em}

% indentsection style, used for sections that aren't already in lists
% that need indentation to the level of all text in the document
\newenvironment{indentsection}[1]%
{\begin{list}{}%
	{\setlength{\leftmargin}{#1}}%
	\item[]%
}
{\end{list}}

% opposite of above; bump a section back toward the left margin
\newenvironment{unindentsection}[1]%
{\begin{list}{}%
	{\setlength{\leftmargin}{-0.5#1}}%
	\item[]%
}
{\end{list}}

% format two pieces of text, one left aligned and one right aligned
\newcommand{\headerrow}[2]
{\begin{tabular*}{\linewidth}{l@{\extracolsep{\fill}}r}
	#1 &
	#2 \\
\end{tabular*}}

% make "C++" look pretty when used in text by touching up the plus signs
\newcommand{\CPP}
{C\nolinebreak[4]\hspace{-.05em}\raisebox{.22ex}{\footnotesize\bf ++}}

% and the actual content starts here
\begin{document}

\begin{center}
{\LARGE \textbf{Ming-Yu Liu}}

(301) 646-8260\ \ \textbullet
\ \ sean.mingyu.liu@gmail.com\ \ \textbullet http://mingyuliu.net/
\\
https://www.linkedin.com/in/mingyuliu\ \ \textbullet 
\href{https://scholar.google.com/citations?hl=en&user=y-f-MZgAAAAJ&cstart=20&pagesize=20}{Google Scholar Profile}\ \ \textbullet 
\href{https://github.com/mingyuliutw}{GitHub}
\ \ 
\end{center}

\hrule
\vspace{0.2em}
% \subsection*{Research}
$\quad$\\
{\bf Objectives:} Seeking a research scientist position in an industrial artificial intelligence research lab or product development department\\
{\bf Research interest:} Computer vision, deep unsupervised learning, deep reinforcement learning\\
{\bf Expertise:} Computer vision, deep learning, and machine learning\\\vspace{-2mm}


% %%%%%%%%%%%%%%%%%%%%%%%%%%%%%%%%%%%%%%%%%%%%%%%%%%%%%%%%%%%%%%%%%%%%%%%%%%%%%%%%%%%%%%%%%%%%%%%%%%%%%%%%%%%%%%%%%%%%%%%%%
\hrule
\vspace{-0.4em}
\subsection*{Education}
\begin{itemize}
	\parskip=0.1em
	\item
	\headerrow
		{\textbf{University of Maryland College Park, Maryland}}
		{\textbf{College Park, MD, USA}}
	\headerrow
		{\emph{Electrical and Computer Engineering, Ph.D.}}
		{\emph{2006 -- 2012}}
		Dissertation: Discrete optimization methods for segmentation and matching\\
		Adviser: Rama Chellappa
	\item
	\headerrow
		{\textbf{National Chiao Tung University}}
		{\textbf{Hsinchu, Taiwan}}
	\headerrow
		{\emph{Electrical Engineering, B.A.}}
		{\emph{1999 -- 2003}}
\end{itemize}
% %%%%%%%%%%%%%%%%%%%%%%%%%%%%%%%%%%%%%%%%%%%%%%%%%%%%%%%%%%%%%%%%%%%%%%%%%%%%%%%%%%%%%%%%%%%%%%%%%%%%%%%%%%%%%%%%%%%%%%%%%
\hrule
\vspace{-0.4em}
\subsection*{Professional Experiences}
\begin{itemize}
	\parskip=0.1em
	\item
	\headerrow
		{\textbf{Mitsubishi Electric Research Laboratories (MERL)}}
		{\textbf{Cambridge, MA, USA}}
	\\
	\headerrow
		{\emph{Principal Research Scientist}}
		{\emph{2012 -- present}}
	\begin{itemize*}
		\item Conducted fundamental and applied research in computer vision and deep learning.
		\item Applied fields: autonomous driving, factory automation, social infrastructure monitoring, and satellite image analysis
    \item Computer vision expertise: object detection, semantic segmentation and labeling, pose estimation, image classification, domain adaptation, depth super-resolution
    \item Deep learning expertise: deep convolutional neural nets, deep generative adversarial nets, attention mechanism and recurrent neural nets, recursive context propagation nets    
		\item Published 10 high impact scientific papers
		\item Earned 5 US patents
		\item Product launched: \href{http://www.mitsubishielectric.com/news/2014/0806.html}{MELFA-3D vision system}
	\end{itemize*}
	\item
	\headerrow
		{\textbf{Intel}}
		{\textbf{Taipei, Taiwan}}
	\\
	\headerrow
		{\emph{Software Engineering Intern}}
		{\emph{2005 -- 2006}}
		Intel X-Scale ARM-based embedded system software development for smart TV applications
	\item
	\headerrow
		{\textbf{Army}}
		{\textbf{Taiwan}}
	\headerrow
		{\emph{Paratrooper Platoon Leader, Military Rank: Second Lieutenant}}
		{\emph{2003 -- 2005}}
\end{itemize}
% %%%%%%%%%%%%%%%%%%%%%%%%%%%%%%%%%%%%%%%%%%%%%%%%%%%%%%%%%%%%%%%%%%%%%%%%%%%%%%%%%%%%%%%%%%%%%%%%%%%%%%%%%%%%%%%%%%%%%%%%%
\hrule
\vspace{-0.4em}
\subsection*{Earned Patents}
\begin{itemize*}
% \item US8428363, US8983177, US8908913, US9195904, US9280827 \vspace{-2mm}
\item US 8,428,363: Method for segmenting images using superpixels and entropy rate clustering
\item US 8,983,177: Method for increasing resolutions of depth images
\item US 8,908,913: Voting-based pose estimation for 3D sensors
\item US 9,195,904: Method for detecting objects in stereo images
\item US 9,280,827: Method for determining object poses using Weighted Features
\end{itemize*}
% %%%%%%%%%%%%%%%%%%%%%%%%%%%%%%%%%%%%%%%%%%%%%%%%%%%%%%%%%%%%%%%%%%%%%%%%%%%%%%%%%%%%%%%%%%%%%%%%%%%%%%%%%%%%%%%%%%%%%%%%%

\hrule
\vspace{-0.4em}
\subsection*{Awards}
\begin{itemize}
\item Best paper honorable mention by Robotics: Science and System Conference RSS, 2015\vspace{-2mm}
\item R\&D 100 Award by R\&D magazine, 2014\vspace{-2mm}
\item University of Maryland College Park, Fellowship, 2011
\end{itemize}
\hrule
\clearpage

\hrule
\vspace{-0.4em}
\subsection*{Publications}
\begin{itemize}
\item {\bf Gaussian Conditional Random Field Network for Semantic Segmentation}\\
      R. Vemulapalli, O. Tuzel, Ming-Yu Liu, R. Chellappa, CVPR 2016\vspace{-2mm}
\item {\bf Deep Gaussian Conditional Random Field Network: A Model-based Deep Network for Denoising}\\
      R. Vemulapalli, O. Tuzel, Ming-Yu Liu, CVPR 2016\vspace{-2mm}
\item {\bf Learning to Remove Multipath Distortions in Time-of-Flight Range Images for a Robotic Arm Setup}\\
      K. Son, Ming-Yu Liu, Y. Taguchi, ICRA 2016\vspace{-2mm}
\item {\bf Unsupervised Network Pretraining via Encoding Human Design}\\
      Ming-Yu Liu, Arun Mallya, Oncel Tuzel, Xi Chen, WACV 2016\vspace{-2mm}
\item {\bf Layered Interpretation of Street View Images}\\
      Ming-Yu Liu, S. Lin, S. Ramalingam, O. Tuzel, RSS 2015 ({\bf Best paper honorable mention})\vspace{-2mm}
\item {\bf Recursive Context Propagation Network for Semantic Scene Labeling}\\
			A. Sharma, O. Tuzel, {Ming-Yu Liu}, {NIPS} 2014\vspace{-2mm}
\item {\bf Learning to Rankd 3D Features}\\
			O. Tuzel, {Ming-Yu Liu}, Y. Taguchi, A. Raghunathan, {ECCV} 2014\vspace{-2mm}
\item {\bf Joint Geodesic Upsampling of Depth Images}\\      
      {Ming-Yu Liu}, O. Tuzel, Y. Taguchi, {CVPR} 2013\vspace{-2mm}
\item {\bf Cluster Analysis via Maximizing a Submodular Function subject to a Matroid Constraint}\\
      {Ming-Yu Liu}, O. Tuzel, S. Ramalingam, R. Chellappa, {TPAMI} 2014\vspace{-2mm}
\item {\bf Model-Based Vehicle Pose Estimation and Tracking in Videos Using Random Forests}\\
      M. Hödlmoser, B. Micusik, M. Pollefeys, {Ming-Yu Liu}, M. Kampel, {3DV} 2013\vspace{-2mm}
\item {\bf Fast Object Detection and Pose Estimation in Heavy Clutter for Robotic Bin-Picking}\\      
			{Ming-Yu Liu}, O. Tuzel, A. Veeraraghavan, Y. Taguchi, T. Marks, R. Chellappa, {IJRR} 2012\vspace{-2mm}
\item {\bf Voting-Based Pose Estimation for Robotic Assembly Using a 3D Sensor}\\
      C. Choi, Y. Taguchi, O. Tuzel, {Ming-Yu Liu}, S. Ramalingam, {ICRA} 2012\vspace{-2mm}
\item {\bf A Grassmann Manifold-based Domain Adaptation Approach}\\
			J. Zheng, {Ming-Yu Liu}, R. Chellappa, P. Phillips, {ICPR} 2012\vspace{-2mm}
\item {\bf Classification and Pose Estimation of Vehicles in Videos by 3D Modeling}\\% within Discrete-Continuous Optimization}\\
      M. Hödlmoser, B. Micusik, {Ming-Yu Liu}, M. Pollefeys, M. Kampel, {3DV} 2012\vspace{-2mm}
\item {\bf Entropy Rate Superpixel Segmentation}\\                
			{Ming-Yu Liu}, O. Tuzel, S. Ramalingam, R. Chellappa, CVPR 2011\vspace{-2mm}
\item {\bf Fast Directional Chamfer Matching}\\                
			{Ming-Yu Liu}, O. Tuzel, A. Veeraraghavan, R. Chellappa, CVPR 2010\vspace{-2mm}
\item {\bf Pose Estimation in Heavy Clutter using a Multi-Flash Camera}\\
      {Ming-Yu Liu}, O. Tuzel, A. Veeraraghavan, R. Chellappa, A. Agrawal, H. Okuda, ICRA 2010\vspace{-2mm}
\end{itemize}
\hrule
\vspace{-0.4em}
\subsection*{Services}
\begin{itemize}
\item {\bf Reviewer}: IEEE TIP, IEEE SPL, CVIU\vspace{-2mm}
\item {\bf Technical committee}: CVPR, ICCV, ECCV, NIPS, ICRA, AAAI\vspace{-2mm}
\end{itemize}
% %%%%%%%%%%%%%%%%%%%%%%%%%%%%%%%%%%%%%%%%%%%%%%%%%%%%%%%%%%%%%%%%%%%%%%%%%%%%%%%%%%%%%%%%%%%%%%%%%%%%%%%%%%%%%%%%%%%%%%%%%
\hrule
\vspace{-0.4em}
\subsection*{Programming Skills}
\begin{indentsection}{\parindent}
\hyphenpenalty=1000
\begin{description*}
	\item[Programming Languages:]
	\CPP, Python, Matlab
	\item[Libraries:]
    Caffe, OpenCV, EIGEN, OpenGL, Coin-OR, GUROBI
	\item[Opensource Code:] $\quad$
	\begin{itemize*}
    \item \href{https://github.com/mingyuliutw/fdcm.git}{Fast directional chamfer matching algorithm}
    \item \href{https://github.com/mingyuliutw/ers.git}{Entropy rate superpixel segmentation algorithm}
    \item \href{http://www.merl.com/research/license/}{Joint geodesic depth upsampling algorithm}
  \end{itemize*}
\end{description*}
\end{indentsection}
% %%%%%%%%%%%%%%%%%%%%%%%%%%%%%%%%%%%%%%%%%%%%%%%%%%%%%%%%%%%%%%%%%%%%%%%%%%%%%%%%%%%%%%%%%%%%%%%%%%%%%%%%%%%%%%%%%%%%%%%%%
\hrule
\vspace{-0.4em}
\subsection*{References}
\begin{indentsection}{\parindent}
\hyphenpenalty=1000
\begin{itemize}
	\item Upon request
\end{itemize}
\end{indentsection}
\hrule
% %%%%%%%%%%%%%%%%%%%%%%%%%%%%%%%%%%%%%%%%%%%%%%%%%%%%%%%%%%%%%%%%%%%%%%%%%%%%%%%%%%%%%%%%%%%%%%%%%%%%%%%%%%%%%%%%%%%%%%%%%
% \hrule
% \vspace{-0.4em}
% \subsection*{Hobbies}
% \begin{itemize}
% \item Snowboarding, Tennis, Basketball, Chinese Kungfu
% \end{itemize}
% %%%%%%%%%%%%%%%%%%%%%%%%%%%%%%%%%%%%%%%%%%%%%%%%%%%%%%%%%%%%%%%%%%%%%%%%%%%%%%%%%%%%%%%%%%%%%%%%%%%%%%%%%%%%%%%%%%%%%%%%%

\end{document}

